% !TEX root = ../thesis.tex

\section{Introduction}
\label{sec:introduction}

\subsection{Motivation and Problem Description}
\label{subsec:motivation}
Sample libraries are ubiquitous in modern audio production. Affordable storage
media and fast computers enable artists, composers and sound designers to work
with large collections of audio files - a one terabyte hard drive costs around
50 euros and can store up to two months of uncompressed audio information in CD
quality. At the same time, the internet is full of large sound corpora for
anyone to download. This is especially true for collections of drum sounds,
which are offered for sale by many companies or traded (sometimes without
considerations for any applicable copyright) on various web forums.
Consequently, many users of digital audio software amass vast collections of
these "sample packs", often without ever listening to or being familiar with
all of their contents, as they encounter the \textit{paradox of choice}
\citep{schwartz2004}. The technological development enabling this abundance of
sound samples also presents new and serious challenges to the efficient use of
them: how can such large numbers of files be searched, organized, compared, and
presented to the user?

\bigskip

The present thesis explores a practical approach to the problem of sound corpus
organization, while focussing primarily on drum samples as used by electronic
music producers.
Currently, arguably the most common way in which producers
search through their sample libraries is by using some kind of file browser
(built into their computer's \gls{os} or their \gls{daw} of choice). This
browser presents a list view of all audio files in the current folder and is
typically sorted alphabetically, chronologically or by some other criterion that
is almost definitely not directly related to the sonic content of the file.

\smallskip

At the same time, advances in the field of \gls{mir} have enabled researchers
and developers to extract descriptive information about the \textit{contents}
of a digital audio file for decades now, making it possible to present much
more relevant data to software users. Commercial software tools incorporating
\gls{aca} are slowly becoming available (see Section \ref{subsec:previous_work}
below). Still, the problem of data visualization persists - how can all these
additional dimensions of information be displayed in a way that improves users'
workflows?

\smallskip

In order to find an alternative organizational method to the name-based,
categorical file browser interface described above, we to turn to \gls{ml},
a field that deals with pattern recognition and classification tasks. The
\textit{\gls{som}} algorithm, first introduced by Teuvo Kohonen (see Section
\ref{subsec:som}, \citet{kohonen1990}), is a machine learning algorithm that
performs dimensionality reduction on a set of higher-dimensional input data and
can at the same time be used for data visualization, as its output is often
two-dimensional and can be shown as a regular grid structure. Besides being an
established algorithm that has been extensively evaluated and used in various
applications (see for example \citet[p.1476]{kohonen1990} for an overview), it
offers a number of further advantages that informed our decision to employ it
in this work. It can visualize high-dimensional data sets, while training itself
completely unsupervised. It can organize unnamed audio files, meaning it can
offer up some proposal for a structure without any meta data. Lastly, the
\gls{som} is based on a grid layout, which is an influential, common structure
in electronic music hardware and music technology in general - grids are
ever-present in electronic music studios \citep{adeney2009}.

% We implement the \textit{\gls{som}} algorithm, first
% introduced by Teuvo Kohonen (see Section \ref{subsec:som}, \citet{kohonen1990}),
% and incorporate it into \textit{CataRT}, an existing software for corpus-based
% concatenative synthesis (CBCS, \citet{schwarz2006}). We also develop a larger
% standalone application called \textit{SOM Browser} that uses the \gls{som} as an
% alternative interface for the exploration of a folder of samples. Finally,


\bigskip


\subsection{Aims and Objectives}
\label{subsec:aims}

implement SOM

practical application for sample browsing

offer alternative to established workflow



is SOM too abstract for users?



\subsection{Previous Work}
\label{subsec:previous_work}

CataRT
\citet{schwarz2006}

AudioQuilt
\citet{fried2014}

Coleman: Navigating the Personal Sample Library
\citet{coleman2007}

SOM usage:

Heise: SoundTorch
\citet{heise2008}

Cosi:
auditory modelling and self-organizing neural networks for timbre classification
\citet{cosi1994}

Pampalk:
hierarchical organization and visualization of drum sample libraries
\citet{pampalk2004}
Computational Models [...]
\citet{pampalk2008}

Infinite Drum Machine
\citet{mcdonald2017}

Sononym
\citet{nielsen2018}

Audio Finder
\citet{audiofinder2019}

Atlas
\citet{atlas2018}
