% !TEX root = ../thesis.tex

\section{Introduction}
\label{sec:introduction}

\subsection{Motivation and Problem Description}
\label{subsec:motivation}

cheap storage and computers allow users to have vast amounts of audio files

especially drum samples: free / low cost sample packs are everywhere

how to deal with them? organize them? search them?



current practice is lists and searching through them in file browsers that
don't display too much information, and even then typically only one dimension
in a two-dimensional window

audio content analysis offers possibility for more relevant information to be
extracted

even with additional information, how to display it properly?


why SOM?

- established algorithm

- can visualize high-dimensional data sets

- completely autonomous

- can organize unnamed files without any meta data

- based on grid structure, which is also very relevant in music technology and
electronic music hardware
\citet{adeney2009}



\subsection{Aims and Objectives}
\label{subsec:aims}

implement SOM

practical application for sample browsing

offer alternative to established workflow



is SOM too abstract for users?



\subsection{Previous Work}
\label{subsec:previous_work}

CataRT
\citet{schwarz2006}

AudioQuilt
\citet{fried2014}

Coleman: Navigating the Personal Sample Library
\citet{coleman2007}

SOM usage:

Heise: SoundTorch
\citet{heise2008}

Cosi:
auditory modelling and self-organizing neural networks for timbre classification
\citet{cosi1994}

Pampalk:
hierarchical organization and visualization of drum sample libraries
\citet{pampalk2004}
Computational Models [...]
\citet{pampalk2008}

Infinite Drum Machine

Magenta / NSynth

Sononym

Audio Finder

Atlas
