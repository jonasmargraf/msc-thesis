% !TEX root = ../thesis.tex

\section{Introduction}
\label{sec:introduction}

\subsection{Motivation and Problem Description}
\label{subsec:motivation}
Sample libraries are ubiquitous in modern audio production. Affordable storage
media and fast computers enable artists, producers and sound designers to work
with large collections of audio files - a one terabyte hard drive costs around
50 Euros and can store up to two months of uncompressed audio information in CD
quality. At the same time, the Internet is full of large sound corpora for
anyone to download. This is especially true for collections of drum sounds,
which are offered for sale by many companies or traded (sometimes without
considerations for any applicable copyright) on various web forums.
Consequently, many users of digital audio software amass vast collections of
these ``sample packs'', often without ever listening to or being familiar with
all of their contents, as they encounter the \textit{paradox of choice}
\citep{schwartz2004}. The technological development enabling this abundance of
sound samples also presents new and serious challenges to the efficient use of
them: how can such large numbers of files be searched, organized, compared, and
presented to the user?

\bigskip

The present thesis explores a practical approach to the problem of sound corpus
organization, while focussing primarily on drum samples as used by electronic
music producers. \textit{Sound corpus} in this work denotes a collection of
sound files --- other commonly encountered terms for such collections are sample
library, pack, kit or bundle. Currently, arguably the most common way in which
producers search through their sample libraries is by using some kind of file
browser (built into their computer's \gls{os} or their \gls{daw} of choice).
This browser presents a list view of all audio files in the current folder and
is typically sorted alphabetically, chronologically or by some other criterion
that is almost definitely not directly related to the sonic content of the file.

\smallskip

At the same time, advances in the field of \gls{mir} have enabled researchers
and developers to extract descriptive information about the contents
of a digital audio file for decades now, making it possible to present much
more relevant data to software users. Commercial software tools incorporating
\gls{aca} are slowly becoming available (see Section \ref{subsec:previous_work}
below). Still, the problem of data visualization persists - how can all these
additional dimensions of information be displayed in a way that improves users'
workflows?

\smallskip

In order to find an alternative organizational method to the name-based,
categorical file browser interface described above, we turn to \gls{ml},
a field that deals with pattern recognition and classification tasks. The
\textit{\gls{som}} algorithm, first introduced by Teuvo Kohonen (see Section
\ref{subsec:som}, \citet{kohonen1990}), is a machine learning algorithm that
performs dimensionality reduction on a set of higher-dimensional input data and
can at the same time be used for data visualization, as its output is often
two-dimensional and can be shown as a regular grid structure. Besides being an
established algorithm that has been extensively evaluated and used in various
applications (see for example \citet[p.1476]{kohonen1990} for an overview), it
offers a number of further advantages that informed our decision to employ it
in this work. It can visualize high-dimensional data sets, while training itself
completely unsupervised. It can organize unnamed audio files, meaning it can
offer up some proposal for a structure without any metadata. Lastly, the
\gls{som} is based on a grid layout, which is an influential, common structure
in electronic music hardware and music technology in general - grids are
ever-present in electronic music studios \citep{adeney2009}.

\subsection{Previous Work}
\label{subsec:previous_work}
The topic of using audio descriptors (see Section
\ref{subsec:feature_extraction} for further details) for the organization of
sounds in two-dimensional interfaces, often displayed as scatter plots, has been
explored in a variety of previous research. Particularly relevant for the
present work is the software \textit{CataRT} by \citet{schwarz2006}. It allows
realtime corpus-based concatenative synthesis and offers the user an interface
to explore the loaded sound corpus interactively. Two selectable audio
descriptors make up the axes along which sounds are plotted as circles, while
two more descriptors can be mapped to circle color and size. At the same time,
more descriptors are already calculated, but cannot be displayed simultaneously,
which informs our decision to extend \textit{CataRT} with the option of using a
two-dimensional \gls{som} as the interface.

\citet{coleman2007} also uses feature extraction to create a scatter plot
interface of a \textit{personal sample library}, which directly relates to our
use case as described in Section \ref{subsec:motivation} above. The author
also implements a way to filter samples by feature values using
\textit{dynamic queries} through realtime interaction with user interface
elements such as sliders and check boxes.

An early example of using \glspl{som} for audio data is \citet{cosi1994}, who
use the algorithm to classify classical musical instruments. They are able to
distinguish between twelve instrumental timbres and also show that new timbres
not used during training can be classified by the map, which will place the
unknown timbre close to the most similar training timbre.

\smallskip

Dealing explicitly with drum samples, \citet{pampalk2004} use a combination of
one- and two-dimensional \glspl{som}. The one-dimensional variant of the
algorithm is used to create a hierarchical sample structure, while a
two-dimensional \gls{som} is used for visualization of the samples.

\smallskip

\citet{heise2008} combine a \gls{som}-based interface with surround sound
reproduction for an interesting and novel approach to sound corpus exploration
with their \textit{SoundTorch} software. The authors let the user carry the
eponymous \textit{SoundTorch}, which is used to illuminate parts of a map
interface onto which sounds are placed. Illuminated sounds are played back on a
multichannel loudspeaker system, with their position in the interface
determining their spatial playback position.

\smallskip

\citet{fried2014} are employing a different approach to visualization and
dimensionality reduction of high-dimensional descriptor spaces with their
\textit{AudioQuilt} software, which relies on user input for its similarity
measure. Their approach achieves a 1-to-1 mapping between sound samples and
grid locations at the cost of not functioning completely unsupervised.

\smallskip

Finally, we can also give a couple of examples of relevant software that exist
outside of academia. The \textit{Infinite Drum Machine} by \citet{mcdonald2017}
uses the t-SNE algorithm \citep{maaten2008} to create a drum machine with a
two-dimensional sound selection interface that resembles a point cloud. A
commercial software tool for organizing and maintaining sound corpora that has
been around for several years now is \textit{Audio Finder}
\citep{audiofinder2019}, which does not employ machine learning at all, instead
offering a large variety of metadata and cataloging tools, as well as some
spectral analysis features. More recently, commercially available audio software
that incorporates machine learning is starting to appear. Two examples of this
are \textit{Sononym} \citep{nielsen2018} and \textit{Atlas} \citep{atlas2018}.
Similarly to \textit{Audio Finder}, \textit{Sononym} deals with sample library
organization and offers a \gls{ml}-based ``similarity search'', while
\textit{Atlas} is a drum sampler plug-in that also features a sound map.

\subsection{Aims and Objectives}
\label{subsec:aims}
The aim of this thesis is to implement the \textit{\gls{som}} algorithm from
scratch and use it to build a software interface for sound corpus organization
and exploration, as there is currently no readily available audio software with
this functionality. To this end, we incorporate the algorithm into
\textit{CataRT}, an existing software for corpus-based concatenative synthesis
(CBCS) \citep{schwarz2006}. We also develop a larger standalone application
called \textit{SOM Browser} that uses the \gls{som} as an alternative interface
for the exploration of a folder of samples, as this is something that no
commercially available software offers. Subsequently, an overview of the effects
of the different controllable parameters of the algorithm on the produced
\gls{som} is given and an ideal map for a representative sound corpus of drum
samples is shown. Finally, we present user feedback to \textit{SOM Browser}
gathered from interviews with five audio professionals. The main question we aim
to explore during these interviews is whether our software can offer a viable
alternative to established workflows or if our proposed alternative interface is
perhaps too abstract to be considered useful.
