% !TEX root = ../thesis.tex

\section{Discussion}
\label{sec:discussion}
This is the Discussion.

Reiterate aims / questions of work



Results interpretation

Able to create application that can produce visual layout of samples with
minimal to no input from user

training length: another improvement after $10^5$ steps, should maybe look at
if this is perceived noticeably better by users?

Why are larger starting neighborhood radii worse than smaller ones? Contradicts
Kohonen. K. mentions sharding / lack of global ordering, could be looked at
(look at soms for data labelled with groups, e.g. subfolders)

Why BDH minimum at m=0?

Why BDH better than Linear? Local learning adaptability...

FNP: adds distortion, but improves map emptiness and distribution of vectors to
nodes, making it a practical usability improvement
naturally, the better the original map is, the less distortion will be added

Interview:

established workflow appears to be "good enough", but nothing more

a clear established workflow could be identified and summarized

interesting is the iterative interplay of mental representation and contextual
evaluation

subjects are aware of limitations of their workflow, bias, etc

SOM Browser: users first impressions were mostly positive, although they were
not able to directly infer meaning to the presented map layout
Some recognize areas of similarity, but the overall impression is that of a
random organization

While preference was with established workflow, the potential of the app was
recognized

Feature requests to improve usability: arrow keys, sample retriggering, larger
font size

Users also would like some more input, through color customization, feature
filters, pre- / sub-categorization of files





Most important results:
- App itself
- FNP: Algorithm extension for optimal utilization of allocated screen space
- Interviews: Identification of existing workflow
- Confirmed need for / interest in alternate interfaces and feedback was
sufficiently positive to warrant further work in this direction



Strengths of the work
- Expert users that build their own MaxMSP performance systems got a new
  addition to their toolbox by expanding mubu-catart with SOM
- simple to use, cross-platform (untested) app that allows no-frills access to
  a ML algorithm not found in any commercially available audio software
- offers, quick, unsupervised visual layout of samples
- can be used for completely unlabelled data



Weaknesses of the work
- reference missing inter-rater reliability,
- potential bias in evaluation results
- subjective coding
- Incomprehensible Map Organization, perhaps due to question phrasing


Limits of the work
- no quantitative results
- limited to one data set, only drum sounds
- SOM Browser is not a tool for very large sample libraries, the used library
pretty much presents the sensible limit of files



What could be done better?
- better feature model to optimize SOM as map of similarity
- Interview: positioning of software as not finished
- Interview: phrasing of question about map organization



Outlook / potential future work

- potential of the application
- Map initialization using PCA
- SOM batch algorithm
- reimagining the software as a plug-in (M4L or as a VST using Juce etc.)






node vector count: 4 corners have tons of samples

4 outliers are the 4 corners

node vector count without FNP: clustering around edges

Reference grounded theory?

Special mention:

Incomprehensible Map Organization

Some recognize areas of similarity, but the overall impression is that of a
random organization.






better feature model for perceived similarity of sounds

limit number of sounds that can be loaded (500?)



\subsection{Outlook}
\label{subsec:outlook}
