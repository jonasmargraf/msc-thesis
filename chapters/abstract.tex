% !TEX root = ../thesis.tex

\begin{abstract}

Large collections of audio files --- \textit{sound corpora} --- have never been
more readily available, as sample libraries are easily accessible online and
cheap storage media effectively eradicate concerns of storage capacity for
contemporary music producers. At the same time, tools for navigating, searching
and organizing these increasingly unmanageable audio file collections have not
kept pace. Currently, arguably the most common tool with which producers search
their sample libraries are file browsers that simply present lists of file names
in alphabetical order.

The present thesis approaches this problem from a practical perspective. We
implement the Self-Organizing Map (SOM), an established machine learning
algorithm for dimensionality reduction and data visualization, and apply it to
sound corpus organization. We present \textit{SOM Browser}, a fast, visual
interface for sample library exploration that offers an alternative to the
established music production workflow by incorporating the SOM algorithm, which
to our knowledge is not available in any commercial audio software. It is a
standalone application that organizes a collection of sound files completely
unsupervised and presents a two-dimensional map of the sounds in an interactive
grid interface with which the user can audition files in rapid succession,
allowing for a quick way to gain an overview of the analyzed sound corpus. To
optimize the space alloted to the map interface, we extend the SOM algorithm
with a new method, which we call Forced Node Population (FNP). FNP reduces
unpopulated (``empty'') areas of the map at the cost of some additional map
distortion. Using a representative sample library of drum sounds, we search for
a set of optimal algorithm parameters according to objective measures of map
quality and produce a map for the chosen sound corpus. We then conduct a series
of qualitative interviews with audio professionals to gain some understanding of
the complex situation that is sample library interaction in a music production
environment and to gauge initial reactions to the alternative software we
developed. Participants' responses allow us to identify a prevalent method of
working with sample libraries, which we codify into a generalized model of the
established workflow. We confirm the need for and interest in alternate
interfaces. Although the organization of sounds in the map interface is seen as
not easily comprehensible, overall interview feedback was sufficiently positive
to warrant further work.

\end{abstract}

\newpage

\begin{otherlanguage}{ngerman}
  \begin{abstract}
    Die Zusammenfassung auch auf Deutsch.
  \end{abstract}
\end{otherlanguage}
\newpage
