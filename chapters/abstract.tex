% !TEX root = ../thesis.tex

\begin{abstract}

Large collections of audio files --- \textit{sound corpora} --- have never been
more readily available. Sample libraries are easily accessible online and
cheap storage media effectively eradicate concerns of storage capacity for
contemporary music producers. At the same time, tools for navigating, searching
and organizing these increasingly unmanageable audio file collections have not
kept pace. At present, arguably the most common tool with which producers search
their sample libraries are file browsers that simply present lists of file names
in alphabetical order.

The present thesis approaches this problem from a practical perspective. We
implement the Self-Organizing Map (SOM), an established machine learning
algorithm for dimensionality reduction and data visualization, and apply it to
sound corpus organization. We present \textit{SOM Browser}, a fast, visual
interface for sample library exploration. It offers an alternative to the
established music production workflow by incorporating the SOM algorithm, which
to our knowledge is not available in any commercial audio software. It is a
standalone application that organizes a collection of sound files completely
unsupervised and presents a two-dimensional map of the sounds. The map forms an
interactive grid interface with which the user can audition files in rapid
succession. This allows for a quick way to gain an overview of the analyzed
sound corpus. To optimize the space alloted to the map interface, we extend the
SOM algorithm with a new method, which we call Forced Node Population (FNP). FNP
reduces unpopulated (``empty'') areas of the map at the cost of some additional
map distortion. Using a representative sample library of drum sounds, we search
for a set of optimal algorithm parameters according to objective measures of map
quality and produce a map for the chosen sound corpus. We then conduct a series
of qualitative interviews with audio professionals to gain some understanding of
the complex situation that is sample library interaction in a music production
environment and to gauge initial reactions to the alternative software we
developed. Participants' responses allow us to identify a prevalent method of
working with sample libraries, which we codify into a generalized model of the
established workflow. The results confirm the need for and interest in alternate
interfaces. Although the organization of sounds in the map interface is seen as
not easily comprehensible, interview responses confirm the need for and interest
in our software.

This work thus presents a functioning proof of principal for the use of SOMs for
sound corpus organization. It demonstrates that there is a high interest in such
methods. Despite interview participants' criticism of details, overall feedback
is positive. Therefore, further development of the presented work in close
exchange with users appears to be very sensible.


% overall interview feedback was sufficiently positive
% to warrant further work.

% Obwohl die Organisation der Klänge in
% unserer Bedienoberfläche als nicht leicht nachvollziehbar eingeschätzt wird,
% bestätigen die erfassten Antworten Bedarf und Interesse an unserer
% Software.
%
% Diese Arbeit legt somit einen funktionierenden \textit{Proof of Principal} für
% die Verwendung von SOMs zur Organisation von Klangkorpora vor. Sie zeigt
% gleichzeitig, dass ein hohes Interesse an solchen Methoden besteht. Trotz
% Detailkritik der Befragten ist das Feedback von Nutzerseite positiv, sodass die
% Weiterentwicklung der präsentierten Arbeit in engem Austausch mit Nutzern mehr
% als sinnvoll erscheint.

\end{abstract}

\newpage

\begin{otherlanguage}{ngerman}
\begin{abstract}

Umfangreiche Sammlungen von Audiodateien --- \textit{Klangkorpora} --- sind so
leicht zugänglich wie nie zuvor. Online verfügbare Sample Libraries und
kostengünstige Datenträger haben Speicherplatz als Einschränkung für moderne
Musikproduzenten effektiv beseitigt. Zugleich hat jedoch das verfügbare
Instrumentarium zur Navigation, Durchsuchung und Organisation dieser zunehmend
unüberschaubaren Datensammlungen nicht Schritt gehalten. Das derzeit wohl
gängigste Werkzeug, mit dem Produzenten ihre Sample Libraries durchsuchen, sind
Dateimanager, welche alphabetisch sortierte Listen von Dateinamen anzeigen.

Die vorliegende Masterarbeit geht dieses Problem praktisch an. Wir
implementieren die Self-Organizing Map (SOM), einen bewährten Algorithmus des
maschinellen Lernens zur Dimensionsreduktion und Datenvisualisierung, und
verwenden diese zur Organisation von Klangkorpora. Wir präsentieren \textit{SOM
Browser}, ein Programm mit einer schnellen, visuellen Bedienoberfläche zur
Erkundung von Sample Libraries. Es bietet eine Alternative zum etablierten
Musikproduktionsprozess, da es den SOM Algorithmus integriert, welcher nach
unserem Wissen in keiner kommerziellen Audiosoftware zur Verfügung steht.
\textit{SOM Browser} ist eine Standalone-Anwendung, welche einen Klangkorpus
komplett unüberwacht organisiert und eine zweidimensionale Karte der Klänge in
einer interaktiven, rasterbasierten Oberfläche präsentiert. Damit können
Klangdateien in schneller Abfolge angespielt werden, was es ermöglicht, rasch
einen Überblick der analysierten Dateien zu bekommen. Um den von der Karte
eingenommenen Platz zu optimieren, erweitern wir den SOM Algorithmus mit einer
neuen Methode, welche wir Forced Node Population (FNP) nennen. FNP reduziert
unbesiedelte (``leere'') Bereiche der Karte auf Kosten zusätzlicher Verzerrung.
Unter Verwendung einer repräsentativen Sample Library von Schlagzeug- und
Perkussionsklängen suchen wir nach einer Reihe von optimalen Parametern des
Algorithmus gemäß objektiver Maße für die Qualität der SOM und erzeugen eine
Karte für den gewählten Klangkorpus. Danach führen wir qualitative Interviews
mit fünf professionellen Produzenten und Künstlern, um ein besseres Verständnis
der Komplexität der Interaktion mit Sample Libraries zu gewinnen und um erste
Reaktionen auf die von uns entwickelte Softwarealternative zu erfassen.
Basierend auf Antworten der Befragten identifizieren wir einen typischen Umgang
mit Sample Libraries, kodifizieren diesen und erstellen ein generalisiertes
Modell der etablierten Arbeitsmethode. Obwohl die Organisation der Klänge in
unserer Bedienoberfläche als nicht leicht nachvollziehbar eingeschätzt wird,
bestätigen die erfassten Antworten Bedarf und Interesse an unserer
Software.

Diese Arbeit legt somit einen funktionierenden \textit{Proof of Principal} für
die Verwendung von SOMs zur Organisation von Klangkorpora vor. Sie zeigt
gleichzeitig, dass ein hohes Interesse an solchen Methoden besteht. Trotz
Detailkritik der Befragten ist das Feedback von Nutzerseite positiv, sodass die
Weiterentwicklung der präsentierten Arbeit in engem Austausch mit Nutzern sehr
sinnvoll erscheint.



% Die erfassten Antworten bestätigen
% außerdem Bedarf und Interesse an alternativen Bedienoberflächen. Obwohl die
% Organisation der Klänge in unserer Software als nicht leicht nachvollziehbar
% eingeschätzt wird, sind die Interviewrückmeldungen hinreichend positiv für
% weiterführende Arbeiten.


\end{abstract}
\end{otherlanguage}
\newpage


% auch wenn details kritisch angemerkt wurden waren ergebnisse so positiv, dass
% sich weitere untersuchungen lohnen
%
% arbeit zeigt erste schritte
%
% proof of principal vorgelegt, technik funktioniert, interesse ist hoch, feedback
% positiv, trotz detailkritik
%
% anwendungsbereich ist bedeutend
%
% hohes interesse gibt
%
% weiter ausbauen
%
% für nutzer optimierte form aufzubereiten
%
% in engem austausch mit nutzern zu entwickeln
