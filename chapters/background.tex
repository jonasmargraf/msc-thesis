% !TEX root = ../thesis.tex

\section{Background}
\label{sec:background}
This is the Background section.

\subsection{Audio Feature Extraction}
\label{subsec:feature_extraction}
% TODO
Make sure to quote \citet{lerch2012}, \citet{rawlinson2015},
\citet{web:meyda2019}, \citet{mathieu2010} \citet{web:yaafe2019}.

\subsubsection{Fundamentals}
\label{subsubsec:feature_fundamentals}

\subsubsection{Audio Pre-Processing}
\label{subsubsec:preprocessing}

\subsubsection{Time-Domain Features}
\label{subsubsec:temporal_features}

\paragraph{Root Mean Square (RMS)}
\label{para:rms}
measures the power of a signal \citep[p.73f]{lerch2012}. It describes sound intensity and is sometimes used as a simple measure for loudness
\citep{web:meyda2019_features} that does not take the nonlinearity of human
hearing into account \citep{fletcher1933}. It is calculated for an audio
frame $x$ consisting of $n$ samples such that
\begin{equation}
  v_{RMS} = \sqrt{ \frac{ \sum_{i=1}^{n} x(i)^2} {n}}.
\end{equation}

\paragraph{Zero-Crossing Rate (ZCR)}
\label{para:zcr}

\subsubsection{Frequency-Domain Features}
\label{subsubsec:spectral_features}

\paragraph{Spectral Centroid}
\label{para:centroid}

\paragraph{Spectral Flatness}
\label{para:flatness}

\paragraph{Spectral Kurtosis}
\label{para:kurtosis}

\paragraph{Spectral Skewness}
\label{para:skewness}

\paragraph{Spectral Slope}
\label{para:slope}

\paragraph{Spectral Spread}
\label{para:spread}

\paragraph{Spectral Rolloff}
\label{para:rolloff}

\subsubsection{Perceptual Features}
\label{perceptual_features}

\paragraph{Loudness}
\label{para:loudness}

\subsection{Self-Organzing Map}
\label{subsec:som}
Something about \glspl{som} and also neurons have \glspl{id}.
