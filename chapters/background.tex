% !TEX root = ../thesis.tex

\section{Background}
\label{sec:background}
This is the Background section.

\subsection{Audio Feature Extraction}
\label{subsec:feature_extraction}

\subsubsection{Fundamentals}
\label{subsubsec:feature_fundamentals}

\subsubsection{Audio Pre-Processing}
\label{subsubsec:preprocessing}

\subsubsection{Time-Domain Features}
\label{subsubsec:temporal_features}

% const featureList = [ 'rms',
%                     'zcr',
%                     'spectralCentroid',
%                     'spectralFlatness',
%                     'spectralSlope',
%                     'spectralRolloff',
%                     'spectralSpread',
%                     'spectralSkewness',
%                     'spectralKurtosis',
%                     'loudness']


\paragraph{Root Mean Square (RMS)}
\label{para:rms}
\gls{rms} goes here.

\paragraph{Zero-Crossing Rate (ZCR)}
\label{para:zcr}

\subsubsection{Frequency-Domain Features}
\label{subsubsec:spectral_features}

\paragraph{Spectral Centroid}
\label{para:centroid}

\paragraph{Spectral Flatness}
\label{para:flatness}

\paragraph{Spectral Kurtosis}
\label{para:kurtosis}

\paragraph{Spectral Skewness}
\label{para:skewness}

\paragraph{Spectral Slope}
\label{para:slope}

\paragraph{Spectral Spread}
\label{para:spread}

\paragraph{Spectral Rolloff}
\label{para:rolloff}

\subsubsection{Perceptual Features}
\label{perceptual_features}

\paragraph{Loudness}
\label{para:loudness}

\subsection{Self-Organzing Map}
\label{subsec:som}
Something about \glspl{som} and also neurons have \glspl{id}.
