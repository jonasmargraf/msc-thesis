% !TEX root = ../thesis.tex

\section{Evaluation}
\label{sec:evaluation}
This is the Evaluation.

\subsection{Sound Corpus Selection}
\label{subsec:eval_corpus_selection}
- which sample library to choose?

- no established reference data set

- DrumEssentials: why? what is it?

\subsection{Measuring SOM-Induced Quantization}
\label{subsec:evaluation_tech}

\subsection{Online Sound Similarity Survey}
\label{subsec:evaluation_survey}

\subsection{Semistructured User Interviews}
\label{subsec:evaluation_interviews}

In order to evaluate the SOM Browser application prototype presented in this
thesis, five semi-structured interviews with working audio professionals were
conducted. These interviews were conducted by the author and consisted of a set
of questions as well as observed user interaction with the prototype software.
For this evaluation, a guide including questions outlining the
structure of the interview as well a set of ratings scales was created. Subjects
were asked about their experience with sample libraries and their current
workflow, and to interact with a sample library in a file browser environment as
well as using the SOM Browser software. Audio from the conversations was
recorded and subsequently analyzed.

\subsubsection{Motivation to Conduct Interviews}
\label{subsubsec:interview_motivation}

This evaluation procedure entails two aspects, namely that of a semi-structured
interview series and a qualitative analysis of the collected responses. The
decision to conduct qualitative interviews stems from the exploratory nature of
the presented work. In order to assess the merit of the developed interface in
its present state, direct feedback from potential users was sought, which
\citet{lazar2017} refers to as "fundamental to human-computer-interaction (HCI)
research" (see \citet[p.187]{lazar2017}). But the motivation for a direct
conversation with users was not only to evaluate the presented interface
proposition, but also for these interviews to serve as an exploration of users'
current situation, to hear about their own experience of it and to see what
advantages and shortcomings they identify in their present workflows. In short,
these interviews were motivated by a desire to gain some understanding of the
complex situation that is sample library interaction in a music production
environment and to gauge initial reactions to the developed prototype
alternative. The semi-structured approach was chosen in order to be able to
react to interviewees' responses more freely and allow the interviewer to ask
follow up questions when deemed necessary. Naturally then, the gathered
responses cannot simply be quantified, which makes a qualitative approach to
their analysis a fitting choice.

\smallskip

There are of course downsides to the chosen approach as well. Conducting
interviews is time-consuming, as it has to be done on a one-on-one basis and
often (as in the case of this work) in person. After the interview is over,
additional time and effort goes into transcribing and annotating the responses.
This severely limits the number of participants that can feasible be recruited
for a study, as is evident by the small number of five participants here.
\citet{lazar2017} identifies another disadvantage of interviews: "[...] data
collection that is separated from the task and context under consideration [...]
suffer[s] from problems of recall. [...] [I]t is, by definition, one step
removed from reality" \citep[p.188ff.]{lazar2017}. Because of this, we follow
the authors' suggestion of combining the interview with user observation.

\bigskip


\subsubsection{Interview Subject Selection}
\label{subsubsec:subject_selection}

- work is not a general tool, aims at specialized users working in music
production
- in order to increase validity and relevance of responses, decision made
to interview working professionals, not hobbyists
- potential future users of this work
- recruited by inquiring in the wider circle
of acquaintances of the author about qualified candidates
- no compensation offered, only sparse information given about nature of
research

\subsubsection{Informed Consent Form}
\label{subsubsec:consent_form}

to document participants agreement to be interviewed
outlines basic information about the interview and its duration, what data will
be collected, data anonymization, interviewee's rights
based on template from TU's ethics commission \citep{web:ethics2019}, can be
found in [appendix]

\subsubsection{Test Subject Code Design}
\label{subsubsec:subject_code}

Responses codified directly at interview using code that can't be reconstructed
by interviewer
Because no interest in matching responses to interviewed individuals
Adapted from a template by TU's ethics commission \citep{web:ethics2019}

\bigskip
selection of main questions

\bigskip
selection of appropriate ratings scale
- 6 vs 5 point Likert scales
- use of appropriate qualifiers

\bigskip
three part structure:
- current workflow
- sample library interaction
- SOM Browser interaction

general question structure: rating, then elaboration

\bigskip
process for categorization of responses (maybe in Results)
