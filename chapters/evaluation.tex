% !TEX root = ../thesis.tex

\section{Evaluation}
\label{sec:evaluation}
This is the Evaluation.

\subsection{Sound Corpus Selection}
\label{subsec:eval_corpus_selection}
- which sample library to choose?

- no established reference data set

- DrumEssentials: why? what is it?

\subsection{Measuring SOM-Induced Quantization}
\label{subsec:evaluation_tech}

\subsection{Online Sound Similarity Survey}
\label{subsec:evaluation_survey}

\subsection{Semistructured User Interviews}
\label{subsec:evaluation_interviews}

In order to evaluate the SOM Browser application prototype presented in this
thesis, five semi-structured interviews with working audio professionals were
conducted. These interviews were conducted by the author and consisted of a set
of questions as well as observed user interaction with the prototype software.
For this evaluation, a guide including questions outlining the
structure of the interview as well a set of ratings scales was created. Subjects
were asked about their experience with sample libraries and their current
workflow, and to interact with a sample library in a file browser environment as
well as using the SOM Browser software. Audio from the conversations was
recorded and subsequently analyzed.

\subsubsection{Motivation to Conduct Interviews}
\label{subsubsec:interview_motivation}

This evaluation procedure entails two aspects, namely a semi-structured
interview series and a qualitative analysis of the collected responses. The
decision to conduct qualitative interviews stems from the exploratory nature of
the presented work. In order to assess the merit of the developed interface in
its present state, direct feedback from potential users was sought, which
\citet{lazar2017} refers to as "fundamental to human-computer-interaction (HCI)
research" (see \citet[p.187]{lazar2017}). But the motivation for a direct
conversation with users was not only to evaluate the presented interface
proposition, but also for these interviews to serve as an exploration of users'
current situation, to hear about their own experience of it and to see what
advantages and shortcomings they identify in their present workflows. In short,
these interviews were motivated by a desire to gain some understanding of the
complex situation that is sample library interaction in a music production
environment and to gauge initial reactions to the developed prototype
alternative. The semi-structured approach was chosen in order to be able to
react to interviewees' responses more freely and allow the interviewer to ask
follow up questions when deemed necessary. Naturally then, the gathered
responses cannot simply be quantified, which makes a qualitative approach to
their analysis a fitting choice.

\smallskip

There are of course downsides to the chosen approach as well. Conducting
interviews is time-consuming, as it has to be done on a one-on-one basis and
often (as in the case of this work) in person. After the interview is over,
additional time and effort goes into transcribing and annotating the responses.
This severely limits the number of participants that can feasible be recruited
for a study, as is evident by the small number of five participants here.
\citet{lazar2017} identifies another disadvantage of interviews: "[...] data
collection that is separated from the task and context under consideration [...]
suffer[s] from problems of recall. [...] [I]t is, by definition, one step
removed from reality" \citep[p.188ff.]{lazar2017}. Because of this, we follow
the authors' suggestion of combining the interview with user observation.

\bigskip


\subsubsection{Interview Subject Selection}
\label{subsubsec:subject_selection}
The SOM Browser application is not aimed at the general population. Instead, it
has been designed for specialized users that work in modern music production, as
they constitute the potential future user base of an application like the one
presented here.

\smallskip

In order to increase the validity and relevance of potential subjects'
responses, the decision was made to interview only working professionals for
this evaluation and to not include hobbyists or people without any experience in
music production.

\smallskip

Subjects were recruited by inquiring about qualified candidates (in other words,
people working professionally in modern music production) in the wider circle of
acquaintances of the author. No compensation was offered and only sparse
information about the nature of the research was given beforehand in order to
minimize the possibility of instilling biases in subjects. Most importantly,
subjects were asked to participate in an interview about sample library
organization, but were not told that they would be shown software developed by
the author.

\subsubsection{Informed Consent Form}
\label{subsubsec:consent_form}
For the purpose of documenting participants agreement to be interviewed, an
informed consent form was created for the interview series. This document
outlines basic information about the purpose and content of the interview and
its duration. It also lists all data that will be collected and explains the
procedure used for data anonymization in order to protect subjects' privacy.
Lastly, it informs participants of their rights to withdraw their consent to the
usage of their data for research purposes and have it erased. This form was
based on a template provided by the ethics commission of \gls{tu-berlin} on
their website \citep{web:ethics2019}. The form used by the author can be found
in
% TODO
XXX REF APPENDIX HERE XXX.

\subsubsection{Test Subject Code Design}
\label{subsubsec:subject_code}
To ensure proper data anonymization, a test subject code was used. This code
is comprised of a series of letters and numbers and was created at the beginning
of the interview by the subjects themselves according to a set of instructions.
All data and responses of the subjects were directly labelled with this code, so
that individuals' names were never used. This code design procedure was again
based on a template by the ethics commission of \gls{tu-berlin} and can be found
on the same website as the information concerning consent forms
\citep{web:ethics2019}. The instruction sheet that was distributed to subjects
can be found in
% TODO
XXX REF APPENDIX HERE XXX.

\subsubsection{Interview Structure}
\label{subsubsec:interview_structure}
The guide developed for this interview can be found in
% TODO
XXX REF APPENDIX HERE XXX
It outlines a three part structure: first, some general questions about
subjects' usage of sample libraries. Second, some guided interaction with a
predetermined sample library in a traditional file browser structure on a
computer. In the third section, the SOM Browser application is finally
introduced and subjects are asked to use it and describe their impression of it.

\subsubsection{Question Design}
\label{subsubsec:question_design}
The general composition employed for most questions is twofold, combining
closed- and open-ended approaches: first,
participants are asked to give a rating on a predefined scale (see
\ref{subsubsec:ratings_scales} below).
Then, participants are free to elaborate on their answer and explain their
rating. If they don't initiate this themselves, a follow-up question along the
lines of "Could you tell me why you chose this rating?" is asked.

\subsubsection{Selection of Ratings Scales}
\label{subsubsec:ratings_scales}
In order to record subjects' ratings, 6 point Likert scales were used (as is
common in \gls{hci} research, see \citet[p.31, p.93]{lazar2017}). The difference
between even and uneven anchor counts in Likert scales lies in the presence (in
the case of uneven anchor counts) or lack (for even counts) of a "neutral"
middle option. Choosing scales without neutral mid-points was motivated by a
desire to encourage subjects to make a definite choice with regard to their
rating. For a short look at the effects of eliminating the mid-point, see
\citet{garland1991}.
\smallskip
The scales presented to subjects were explicitly labeled textually instead of
numerically. The anchor points were designed using two polar adjectives (such as
"positive" and "negative") and a consistent, three-tiered set of adjective
qualification with "very" marking the strongest option, followed by the
adjective without qualifier and then "somewhat" as the weakest variant. The
resulting scale for a positive/negative rating is composed of the following
anchors: very positive, positive, somewhat positive, somewhat negative,
negative, very negative. The selection of these qualifiers and appropriate
anchors in general was inspired partially by \citet{vagias2006}. The full set of
scales used for the conducted interviews can be found in
% TODO
XXX REF APPENDIX HERE XXX.

\subsubsection{Questions Used}
\label{subsubsec:questions_used}
In section 1, which serves as an introduction for the interviewee, general
administrative requirements such as the signing of the consent form and a
topical introduction of the research are taken care off. This is then followed
by two simple Yes/No questions to establish whether the subject works with
third-party and personally created sample library (see questions 1.1. and 1.2.).

\smallskip

Section 2 begins with a presentation of the \textit{Drum Essentials} sample
library to the subject. This presentation includes the information that it is a
library of drum samples that consists of around 1000 sound files which are
organized in subfolders according to the respective instrument, such as kick
drum, snare drum, hi-hat, and so forth. The interviewee is invited to explore
the sample library using the laptop that it is being presented on. Then, in
question 2.1., subjects are asked to describe how to approach familiarizing
themselves with the provided sample library in order to use its contents in a
hypothetical work project of theirs. Question 2.2. follows this up with a
request for a rating of the subject's level of satisfaction with the workflow
that they outlined.

\smallskip

In the third and final section of the interview, the SOM Browser software is
introduced to participants.


\bigskip
process for categorization of responses (maybe in Results)
