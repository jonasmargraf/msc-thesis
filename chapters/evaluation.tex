% !TEX root = ../thesis.tex

\section{Evaluation}
\label{sec:evaluation}
This is the Evaluation.

\subsection{Sound Corpus Selection}
\label{subsec:eval_corpus_selection}
- which sample library to choose?

- no established reference data set

- DrumEssentials: why? what is it?

\subsection{Measuring SOM-Induced Quantization}
\label{subsec:evaluation_tech}

\subsection{Online Sound Similarity Survey}
\label{subsec:evaluation_survey}

\subsection{Semistructured User Interviews}
\label{subsec:evaluation_interviews}

In order to evaluate the SOM Browser application prototype presented in this
thesis, 5 semi-structured interviews with working audio professionals were
conducted. These interviews were conducted by the author and consisted of a set
of questions as well as observed user interaction with the prototype software.
For this evaluation, a guide including questions outlining the
structure of the interview as well a set of ratings scales was created. Subjects
were asked about their experience with sample libraries and their current
workflow, and to interact with a sample library in a file browser environment as
well as using the SOM Browser software. Audio from the conversations was
recorded and subsequently analyzed.

\subsubsection{Motivation to Conduct Interviews}
\label{subsubsec:interview_motivation}

why qualitative evaluation / interviews?

- direct user feedback
- given exploratory nature of this work, decision to "go deep but not
broad"
- initial exploration about users current situation and their assessment
of it and potential problems, shortcomings
- gain understanding of complex situation
- combination of interview and observation
- desire to gauge initial reactions about prototype

\bigskip


\subsubsection{Interview Subject Selection}
\label{subsubsec:subject_selection}

- work is not a general tool, aims at specialized users working in music
production
- in order to increase validity and relevance of responses, decision made
to interview working professionals, not hobbyists
- potential future users of this work
- recruited by inquiring in the wider circle
of acquaintances of the author about qualified candidates
- no compensation offered, only sparse information given about nature of
research

\subsubsection{Informed Consent Form}
\label{subsubsec:consent_form}

to document participants agreement to be interviewed
outlines basic information about the interview and its duration, what data will
be collected, data anonymization, interviewee's rights
based on template from TU's ethics commission \citep{web:ethics2019}, can be
found in [appendix]

\subsubsection{Test Subject Code Design}
\label{subsubsec:subject_code}

Responses codified directly at interview using code that can't be reconstructed
by interviewer
Because no interest in matching responses to interviewed individuals
Adapted from a template by TU's ethics commission \citep{web:ethics2019}

\bigskip
selection of main questions

\bigskip
selection of appropriate ratings scale
- 6 vs 5 point Likert scales
- use of appropriate qualifiers

\bigskip
three part structure:
- current workflow
- sample library interaction
- SOM Browser interaction

general question structure: rating, then elaboration

\bigskip
process for categorization of responses (maybe in Results)
